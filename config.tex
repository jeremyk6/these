% Paquets utilisés
\usepackage[french]{babel}
\usepackage[utf8]{inputenc}
\usepackage{graphicx}
\usepackage{xcolor}
\usepackage[hybrid]{markdown}
\usepackage{fancyhdr}
\usepackage[style=alphabetic]{biblatex}
\usepackage{hyperref}
\usepackage{pdfcomment}

% Défini le format des pages (A4, marges)
\usepackage[a4paper,width=160mm,top=25mm,bottom=25mm]{geometry}

% Style avec le bandeau en haut de page
 
\fancypagestyle{main}{
   \fancyhf{}
   \fancyhead[LO]{\nouppercase{\rightmark}}
   \fancyhead[RO,LE]{\thepage}
   \fancyhead[RE]{\nouppercase{\leftmark}}
}

\pagestyle{main}

% Dossier des images
\graphicspath{ {images/} }

% Bibliographie
\addbibresource{bibliographie.bib}

% Commandes de commentaires
% Exemple d'utilisation : \jmf{Voici un commentaire}
\newcommand*{\gt}[1]{\pdfcomment[author=Guillaume, color=red]{#1}} % Guillaume
\newcommand*{\ff}[1]{\pdfcomment[author=Fabien, color=green]{#1}} % Fabien
\newcommand*{\jmf}[1]{\pdfcomment[author=Jean-Marie, color=blue]{#1}} % Jean-Marie

% Commandes personnelles
\newcommand{\todo}[1]{{\color{red}\textbf{TODO} #1}}
\newcommand{\missref}[1]{{\color{blue}\textbf{(référence ?)} #1}}
\newcommand*{\comment}[1]{\pdfcomment[author=Jérémy]{#1}}
\newcommand*{\newpar}[1]{\vspace{.7cm}}

% Commandes de vocabulaire
\newcommand{\pcdv}[1]{personne concernée par la déficience visuelle}
\newcommand{\pcdvs}[1]{personnes concernées par la déficience visuelle}
\newcommand{\ipa}[1]{instructeur pour l'autonomie}
\newcommand{\ipas}[1]{instructeurs pour l'autonomie}