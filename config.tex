% Paquets utilisés
\usepackage[french]{babel}
\usepackage[utf8]{inputenc}
\usepackage{graphicx}
\usepackage{xcolor}
\usepackage{fancyhdr}
\usepackage[numbers]{natbib}
\usepackage{hyperref}
\usepackage{pdfcomment}
\usepackage{glossaries}
\usepackage{lipsum}
\usepackage{array}
\usepackage{csquotes}
\usepackage{amssymb}
\usepackage{fontawesome}
\usepackage{plantuml}
\usepackage{subcaption}
\usepackage{listings}
\usepackage{makecell}
\usepackage{algorithm}
\usepackage{algpseudocode}
\usepackage{pgf-pie}
\usepackage{soulutf8}
\usepackage{appendix}
%algorithm2e (francisable)

% Alignement des cellules makecell à gauche
\renewcommand{\cellalign}{tl}
\renewcommand{\theadalign}{tl}

% Francisation des commandes algorithmiques
\algrenewcommand\algorithmicfor{\textbf{pour}}
\algrenewcommand\algorithmicdo{\textbf{faire}}
\algrenewcommand\algorithmicend{\textbf{fin}}

% Défini le format des pages (A4, marges)
\usepackage[a4paper,width=160mm,top=25mm,bottom=25mm]{geometry}

% Défini la profondeur de numérotation
\setcounter{secnumdepth}{4}

% Style avec le bandeau en haut de page
 
\fancypagestyle{main}{
   \fancyhf{}
   \fancyhead[LO]{\nouppercase{\rightmark}}
   \fancyhead[RO,LE]{\thepage}
   \fancyhead[RE]{\nouppercase{\leftmark}}
}

\pagestyle{main}

% Dossier des images
\graphicspath{ {images/} }

% Bibliographie
\bibliographystyle{unsrtnat}

% Commandes de commentaires
% Exemple d'utilisation : \jmf{Voici un commentaire}
\newcommand*{\gt}[1]{\pdfcomment[author=Guillaume, color=red]{#1}} % Guillaume
\newcommand*{\ff}[1]{\pdfcomment[author=Fabien, color=green]{#1}} % Fabien
\newcommand*{\jmf}[1]{\pdfcomment[author=Jean-Marie, color=blue]{#1}} % Jean-Marie

% Commandes personnelles
\newcommand{\todo}[1]{{\color{red}\textbf{TODO} #1}}
\newcommand{\missref}[1]{{\color{blue}\textbf{(référence ?)} #1}}
\newcommand{\addref}[1]{{\color{red}(#1)}}
\newcommand{\missdef}[1]{{\color{yellow}\textbf{(définition ?)} #1}}
\newcommand{\pquote}[1]
\newcommand*{\comment}[1]{\pdfcomment[author=Jérémy]{#1}}
\newcommand*{\addfigure}[1]{\pdfcomment[author=Jérémy, icon=Note]{#1}}
\newcommand*{\newpar}[1]{\vspace{.3cm}}


% Commandes de vocabulaire
\newcommand{\pcdv}[1]{personne concernée par la déficience visuelle}
\newcommand{\pcdvs}[1]{personnes concernées par la déficience visuelle}
\newcommand{\ipa}[1]{instructeur pour l'autonomie}
\newcommand{\ipas}[1]{instructeurs pour l'autonomie}
\newcommand{\adt}[1]{adaptateur-transcripteur}
\newcommand{\adts}[1]{adaptateurs-transcripteurs}

% Commandes de mise en forme
\newcommand{\osmkey}[1]{\faKey\texttt{~#1}}
\newcommand{\osmvalue}[1]{\faFileTextO\texttt{~#1}}
\newcommand{\hlc}[2][yellow]{{\sethlcolor{#1}\hl{#2}}}

% Définitions
\newtheorem{definition}{Définition} 

% Glossaire
\newacronym{sig}{SIG}{Système d'Information Géographique}
\newacronym{sgbdr}{SGBDR}{Système de Gestion de Bases de Données Relationnelles}
\newacronym[plural=PCDV, firstplural=Personnes Concernées par la Déficience Visuelle]{pcdv}{PCDV}{Personne Concernée par la Déficience Visuelle}
\newacronym{cerema}{CEREMA}{Centre d'Études et d'expertise sur les Risques, l'Environnement, la Mobilité et l'Aménagement}
\newacronym{bev}{BEV}{Bandes d'Éveil de Vigilance}
\newacronym{igv}{IGV}{Information Géographique Volontaire}
\newacronym{gtfs}{GTFS}{General Transit Feed Specification}
\newacronym{netex}{NeTEx}{Network Timetable Exchange}
\newacronym{lom}{LOM}{Loi d'Orientation des Mobilités}
\newacronym{cnig}{CNIG}{Conseil National de l'Information Géolocalisée}
\newacronym{osm}{OSM}{OpenStreetMap}
\newacronym[plural=GPS, firstplural=Global Positioning Systems]{gps}{GPS}{Global Positioning System}
\newacronym{gnss}{GNSS}{Global Navigation Satellite Systems}
\newacronym{afiadv}{AFIADV}{Association Francophone des Instructeurs pour l'Autonomie des personnes Déficientes Visuelles}
\newacronym{aildv}{AILDV}{Association des Instructeurs de Locomotion pour personnes Déficientes Visuelles}
\newacronym{avjadv}{AVJADV}{Association des Instructeurs pour l'Autonomie dans la Vie Journalière des personnes Aveugles ou Déficientes Visuelles}
\newacronym[firstplural=Instructeurs de Locomotion]{il}{IL}{Instructeur de Locomotion}
\newacronym[firstplural=Instructeurs en Autonomie de la Vie Journalière]{iavj}{IAVJ}{Instructeur en Autonomie de la Vie Journalière}
\newacronym[firstplural=Instructeurs pour l'Autonomie des personnes déficientes visuelles]{ia}{IA}{Instructeur pour l'Autonomie des personnes déficientes visuelles}
\newacronym{crdv}{CRDV}{Centre de Rééducation pour Déficients Visuels}
\newacronym[firstplural=Personnes à mobilité réduite]{pmr}{PMR}{Personne à mobilité réduite}
\newacronym{limos}{LIMOS}{Laboratoire d'Informatique, de Modélisation et d'Optimisation des Systèmes}
\newacronym{ign}{IGN}{Institut National de l’Information Géographique et Forestière}
\newacronym{irit}{IRIT}{Institut de Recherche en Informatique de Toulouse}
\newacronym{deri}{DERi}{Document En Relief Interactif}
\makenoidxglossaries