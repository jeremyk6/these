% Paquets utilisés
\usepackage[french]{babel}
\usepackage[utf8]{inputenc}
\usepackage{graphicx}
\usepackage{xcolor}
\usepackage{fancyhdr}
\usepackage[natbib, style=alphabetic]{biblatex}
\usepackage{hyperref}
\usepackage{pdfcomment}
\usepackage[nonumberlist]{glossaries}
\usepackage{lipsum}
\usepackage{array}
\usepackage{csquotes}
\usepackage{amssymb}
\usepackage{fontawesome}
\usepackage{plantuml}
\usepackage{subcaption}
\usepackage{listings}
\usepackage{makecell}
\usepackage{pgf-pie}
\usepackage{soulutf8}
\usepackage{appendix}
\usepackage{float}
\usepackage{minted}
\usepackage{ulem}
\usepackage[french, onelanguage]{algorithm2e} %(francisable)

% Style des citations
\addbibresource{bibliography/bibliographie.bib}
\renewcommand*{\labelalphaothers}{}

% Alignement des cellules makecell à gauche
\renewcommand{\cellalign}{tl}
\renewcommand{\theadalign}{tl}

% Défini le format des pages (A4, marges)
\usepackage[a4paper,width=160mm,top=25mm,bottom=25mm]{geometry}

% Défini la profondeur de numérotation
\setcounter{secnumdepth}{4}

% Style avec le bandeau en haut de page
 
\fancypagestyle{main}{
   \fancyhf{}
   \fancyhead[LO]{\nouppercase{\rightmark}}
   \fancyhead[RO,LE]{\thepage}
   \fancyhead[RE]{\nouppercase{\leftmark}}
}

\pagestyle{main}

% Dossier des images
\graphicspath{ {images/} }

% Commandes de commentaires
% Exemple d'utilisation : \jmf{Voici un commentaire}
\newcommand*{\gt}[1]{\pdfcomment[author=Guillaume, color=red]{#1}} % Guillaume
\newcommand*{\ff}[1]{\pdfcomment[author=Fabien, color=green]{#1}} % Fabien
\newcommand*{\jmf}[1]{\pdfcomment[author=Jean-Marie, color=blue]{#1}} % Jean-Marie

% Commandes personnelles
\newcommand{\todo}[1]{{\color{red}\textbf{TODO} #1}}
\newcommand{\missref}[1]{{\color{blue}\textbf{(référence ?)} #1}}
\newcommand{\addref}[1]{{\color{red}(#1)}}
\newcommand{\missdef}[1]{{\color{yellow}\textbf{(définition ?)} #1}}
\newcommand{\pquote}[1]
\newcommand*{\comment}[1]{\pdfcomment[author=Jérémy]{#1}}
\newcommand*{\addfigure}[1]{\pdfcomment[author=Jérémy, icon=Note]{#1}}
\newcommand*{\newpar}[1]{\vspace{.3cm}}


% Commandes de vocabulaire
\newcommand{\pcdv}[1]{personne concernée par la déficience visuelle}
\newcommand{\pcdvs}[1]{personnes concernées par la déficience visuelle}
\newcommand{\ipa}[1]{instructeur pour l'autonomie}
\newcommand{\ipas}[1]{instructeurs pour l'autonomie}
\newcommand{\adt}[1]{adaptateur-transcripteur}
\newcommand{\adts}[1]{adaptateurs-transcripteurs}

% Commandes de mise en forme
\newcommand{\osmkey}[1]{\faKey\texttt{~#1}}
\newcommand{\osmvalue}[1]{\faFileTextO\texttt{~#1}}
\newcommand{\hlc}[2][yellow]{{\sethlcolor{#1}\hl{#2}}}

% Définitions
\newtheorem{definition}{Définition} 

% Glossaire
\newglossaryentry{sig}{
      name={SIG},
      description={Système d'Information Géographique},
      first={Système d'Information Géographique (SIG)},
      firstplural={Systèmes d'Information Géographique (SIGs)},
}

\newglossaryentry{sgbdr}{
      name={SGBDR},
      description={Système de Gestion de Bases de Données Relationnelles},
      first={Système de Gestion de Bases de Données Relationnelles (SGBDR)},
      firstplural={Systèmes de Gestion de Bases de Données Relationnelles (SGBDRs)},
}

\newglossaryentry{pcdv}{
      name={PCDV},
      description={Personne Concernée par la Déficience Visuelle},
      first={Personne Concernée par la Déficience Visuelle (PCDV)},
      firstplural={Personnes Concernées par la Déficience Visuelle (PCDVs)},
}

\newglossaryentry{cerema}{
      name={CEREMA},
      description={Centre d'Études et d'expertise sur les Risques, l'Environnement, la Mobilité et l'Aménagement},
      first={Centre d'Études et d'expertise sur les Risques, l'Environnement, la Mobilité et l'Aménagement (CEREMA)},
}

\newglossaryentry{bev}{
      name={BEV},
      description={Bande d'Éveil de Vigilance},
      first={Bande d'Éveil de Vigilance (BEV)},
      firstplural={Bandes d'Éveil de Vigilance (BEVs)},
}

\newglossaryentry{igv}{
      name={IGV},
      description={Information Géographique Volontaire},
      first={Information Géographique Volontaire (IGV)},
}

\newglossaryentry{gtfs}{
      name={GTFS},
      description={General Transit Feed Specification},
      first={General Transit Feed Specification (GTFS)},
}

\newglossaryentry{netex}{
      name={NeTEx},
      description={Network Timetable Exchange},
      first={Network Timetable Exchange (NeTEx)},
}

\newglossaryentry{lom}{
      name={LOM},
      description={Loi d'Orientation des Mobilités},
      first={Loi d'Orientation des Mobilités (LOM)},
}

\newglossaryentry{cnig}{
      name={CNIG},
      description={Conseil National de l'Information Géolocalisée},
      first={Conseil National de l'Information Géolocalisée (CNIG)},
}

\newglossaryentry{osm}{
      name={OSM},
      description={OpenStreetMap},
      first={OpenStreetMap (OSM)},
}

\newglossaryentry{gps}{
      name={GPS},
      description={Global Positioning System},
      first={Global Positioning System (GPS)},
      firstplural={Global Positioning Systems (GPS)},
}

\newglossaryentry{gnss}{
      name={GNSS},
      description={Global Navigation Satellite Systems},
      first={Global Navigation Satellite Systems (GNSS)},
}

\newglossaryentry{afiadv}{
      name={AFIADV},
      description={Association Francophone des Instructeurs pour l'Autonomie des personnes Déficientes Visuelles},
      first={Association Francophone des Instructeurs pour l'Autonomie des personnes Déficientes Visuelles (AFIADV)},
}

\newglossaryentry{aildv}{
      name={AILDV},
      description={Association des Instructeurs de Locomotion pour personnes Déficientes Visuelles},
      first={Association des Instructeurs de Locomotion pour personnes Déficientes Visuelles (AILDV)},
}

\newglossaryentry{avjadv}{
      name={AVJADV},
      description={Association des Instructeurs pour l'Autonomie dans la Vie Journalière des personnes Aveugles ou Déficientes Visuelles},
      first={Association des Instructeurs pour l'Autonomie dans la Vie Journalière des personnes Aveugles ou Déficientes Visuelles (AVJADV)},
}

\newglossaryentry{il}{
      name={IL},
      description={Instructeur de Locomotion},
      first={Instructeur de Locomotion (IL)},
      firstplural={Instructeurs de Locomotion (ILs)},
}

\newglossaryentry{iavj}{
      name={IAVJ},
      description={Instructeur en Autonomie de la Vie Journalière},
      first={Instructeur en Autonomie de la Vie Journalière (IAVJ)},
      firstplural={Instructeurs en Autonomie de la Vie Journalière (IAVJs)},
}

\newglossaryentry{ia}{
      name={IPA},
      description={Instructeur Pour l'Autonomie des personnes déficientes visuelles},
      first={Instructeur Pour l'Autonomie des personnes déficientes visuelles (IPA)},
      firstplural={Instructeurs Pour l'Autonomie des personnes déficientes visuelles (IPAs)},
}

\newglossaryentry{crdv}{
      name={CRDV},
      description={Centre de Rééducation pour Déficients Visuels},
      first={Centre de Rééducation pour Déficients Visuels (CRDV)},
}

\newglossaryentry{pmr}{
      name={PMR},
      description={Personne à Mobilité Réduite},
      first={Personne à Mobilité Réduite (PMR)},
      firstplural={Personnes à Mobilité Réduite (PMRs)},
}

\newglossaryentry{limos}{
      name={LIMOS},
      description={Laboratoire d'Informatique, de Modélisation et d'Optimisation des Systèmes},
      first={Laboratoire d'Informatique, de Modélisation et d'Optimisation des Systèmes (LIMOS)},
}

\newglossaryentry{ign}{
      name={IGN},
      description={Institut National de l’Information Géographique et Forestière},
      first={Institut National de l’Information Géographique et Forestière (IGN)},
}

\newglossaryentry{irit}{
      name={IRIT},
      description={Institut de Recherche en Informatique de Toulouse},
      first={Institut de Recherche en Informatique de Toulouse (IRIT)},
}

\newglossaryentry{inja}{
      name={INJA},
      description={Institut National des Jeunes Aveugles},
      first={Institut National des Jeunes Aveugles (INJA)},
}

\newglossaryentry{oms}{
      name={OMS},
      description={Organisation Mondiale de la Santé},
      first={Organisation Mondiale de la Santé (OMS)},
}
\newglossaryentry{deri}{
      name={DERi},
      description={Document En Relief Interactif},
      first={Document En Relief Interactif (DERi)},
      firstplural={Documents En Relief Interactifs (DERis)},
}
\newglossaryentry{activmap}{
      name={ACTIVmap},
      description={Assistance à la Conception de carTes pour défIcients Visuels},
      first={Assistance à la Conception de carTes pour défIcients Visuels (ACTIVmap)}
}
\newglossaryentry{lastig}{
      name={LASTIG},
      description={Laboratoire des Sciences et Technologies de l'Information Géographique},
      first={Laboratoire des Sciences et Technologies de l'Information Géographique (LASTIG)}
}
\makenoidxglossaries