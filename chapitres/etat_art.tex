\chapter{État de l'art}

\section{Se repérer dans l'espace urbain pour une personne déficiente visuelle}

%
% Qu'est-ce que la déficience visuelle ? Comment différencier malvoyants et non-voyants ? Qu'est-ce qu'un résidu visuel ?
%

La déficience visuelle correspond à la perte de tout ou partie de la vue. On quantifie la capacité à discriminer les objets à l'aide de l'indicateur d'acuité visuelle \comment{Et aussi le champ de vision}. Ce dernier est notamment utilisé par l'OMS pour définir la déficience visuelle, et au-delà la malvoyance et la cécité. Ainsi, on parle de malvoyance lorsque l'œil le plus performant a un score de 3/10\up{ème}, et de cécité lorsqu'il a un score inférieur à 1/20\up{ème}. Selon le degré de cécité, il peut subsister un résidu visuel qui correspond à une perception de la lumière ou du contour des objets. 

La sévérité de l'atteinte visuelle affecte la capacité des personnes à se déplacer en autonomie \cite{homere_2023}. Au-delà de l'atteinte visuelle, les handicaps associés peuvent également contribuer à rendre le déplacement plus difficile. En ce sens, les besoins peuvent varier d'une personne à une autre.

\subsection{Les mécanismes cognitifs mobilisés par les PSDV}

%
% Stratégies de déplacement (technique de la canne blanche, chien guide, écholocation)
%

Plusieurs stratégies sont mises en place par les personnes concernées lors de leur déplacement pour compenser l'absence de vision à l'aide des autres sens. La principale concerne la canne blanche, qui joue un rôle double. Elle permet en premier lieu de détecter les obstacles et objets présents sur son chemin. Mais elle joue également un rôle social en signalant aux autres usagers que la personne qui la porte est aveugle ou malvoyante. Par ailleurs, il existe des cannes blanches électroniques, équipées de capteurs pouvant détecter des obstacles à distance. Ces dernières sont cependant moins utilisées que leur variante traditionnelle. 

Mais la canne blanche n'est pas suffisante pour se repérer dans l'espace. Son champ d'action se limite à une surface réduite face à l'utilisateur, au contraire de l'ouïe qui va permettre à la personne d'appréhender l'espace en écoutant les sons de l'environnement urbain. Détecter un moteur à l'arrêt peut par exemple permettre de repérer un passage piéton en aval d'une voiture. Les dispositifs sonores en ville qui équipent certains feux vont permettre à la fois de repérer une traversée, mais également de la réaliser en sécurité. L'ouïe joue également un rôle fort dans l'écholocation, c’est-à-dire la capacité à utiliser les échos pour percevoir des objets. \todo{}

% Au-delà de l'écholocation, parler de proprioception

% Les outils de l'autonomie : carte mentale PCF, solutions techniques type GPS

\newpar{}

En cognition spatiale, une carte mentale correspond à un modèle conceptuel d'un espace et des objets qui le composent \addref{Fernandes, 2019}. Pour se déplacer en autonomie dans un environnement urbain, avoir conscience de la position relative des espaces connus pour pouvoir s'orienter. Un aspect important de cette carte mentale concerne les repères. Ceux-ci vont permettre, à l'aide des notions d'indices et de preuves \addref{Jouffrais, 201X}, de confirmer sa position et sa direction. Les repères peuvent faire appel au sens du toucher (un objet que l'on va sentir à la canne), de l'ouïe (une fontaine en fonctionnement), ou de l'odorat (en boulangerie ouverte). On peut également définir les repères à l'aide de trois attributs : Permanents, Caractéristiques et Fiables (PCF) \comment{Citer le chapitre du livre}. Ainsi, un repère qui respecte ces trois critères sera tout le temps mobilisables, au contraire d'une fontaine ou d'un commerce qui peuvent être intermittents. 

La carte mentale peut être alimentée par les déplacements habituels des personnes qui repèrent l'environnement. Mais elle peut également l'être à l'aide de différents dispositifs. Une carte tactile peut aider à comprendre un espace non familier. Mais les personnes peuvent également utiliser un GPS piéton qui pourra décrire en cours de route l'itinéraire, l'infrastructure et les points d'intérêt croisés sur le chemin.

% Parler des causes de la déficience visuelle (DMLA, accident) ? Pourquoi parle-t-on de handicaps associés (conclusion de la sous-partie) ?

\todo{}

\subsection{Les pratiques des \ipas{}}

\label{pratiques_ipas}

L'Association Francophone des Instructeurs pour l'Autonomie des personnes Déficientes Visuelles (AFIADV) est issue de la fusion de l'Association des Instructeurs de Locomotion pour personnes Déficientes Visuelles (AILDV) et de l'Association des Instructeurs pour l'Autonomie dans la Vie Journalière des personnes Aveugles ou Déficientes Visuelles (AVJADV). Ces dernières formaient respectivement les Instructeurs de Locomotion (IL) et les Instructeurs en Autonomie de la Vie Journalière (IAVJ). L'AFIADV forme aujourd'hui des Instructeurs pour l'Autonomie des personnes Déficientes Visuelles (IA) dont la profession regroupe les deux expertises précédentes. Dans ce manuscrit, nous allons nous intéresser particulièrement à l'instruction en locomotion, aussi appelée instruction en orientation et mobilité. \comment{Indiquer qu'il y a peu de littérature francophone sur les pratiques des IL}

\begin{figure}
    \centering
    \includegraphics{images/placeholder.jpg}
    \caption{Lors d'une séance de locomotion, un \ipa{} réalise parfois une carte de situation en utilisant des aimants. Ce dispositif permet notamment une interaction avec l'apprenant qui peut les déplacer selon sa compréhension du lieu. Pour représenter une zone plus complexe et faire une carte plus exploratoire, les \ipas{} réalisent des cartes imprimées sur du papier thermogonflé. Ces cartes sont aujourd'hui réalisées en décalquant une photographie aérienne et complétées par connaissance du terrain, notamment sur les zones couvertes par la végétation.}
    \label{fig:il_carte}
\end{figure}

Dans ce contexte, le rôle de l'\ipa{} est d'accompagner la personne aveugle ou malvoyante dans la compensation de la perte de la vue par l'usage des autres sens. Elle peut, pour cela, mobiliser des cartes en relief. S'il existe des outils de conception automatisée (voir \ref{ea_cartetactile}), la pratique de réalisation des cartes en relief est aujourd'hui majoritairement artisanale \comment{Erreur insertion figure ?!}.

Pour les carrefours spécifiquement, à l'instar des aménageurs, les \ipas{} mobilisent à la fois les typologies standards et les branches. \todo{}

\section{Les données géographiques et l'accessibilité}

\subsection{Que sont les données d'accessibilité ?}

% Définir les données géographiques et leur type : couche, graphe, etc.

Dans un \gls{sig}, les données sont généralement représentées sous la forme de couches dont l'empilement compose la carte, à la manière des logiciels de dessin assisté par ordinateur (voir \ref{fig:geodatalayers}). On distingue principalement les données raster et les données vectorielles. Les premières correspondent à une grille dont chaque cellule contient une valeur, et permettent de stocker par exemple des images ou des reliefs. Les secondes correspondent à des données structurées ou non auxquelles est accolée une abstraction géométrique de la réalité. Il s'agit principalement de données tabulaires avec une ou plusieurs colonnes de géométrie, ces dernières pouvant être des points, des lignes ou des polygones. Une couche est ainsi une table telle qu'on peut la définir dans un \gls{sgbdr} \cite{AschanLeygonie2019}. Le paradigme en couches n'est pas la seule manière de représenter les données géographiques : il est par exemple possible de les représenter sous forme de graphe \cite{Palacio2003}.

\begin{figure}
    \centering
    \resizebox{\linewidth}{!}{\includegraphics{images/layers.png}}
    \caption{
        Représentation en couches de données géographiques.}
    \label{fig:geodatalayers}
\end{figure}

\newpar{}

% Parler des dispositifs d'accessibilité (BEV, etc., voir CEREMA): pas d'exhaustivité sur les dispositifs, précision plus forte dans Modélisation

Une donnée d'accessibilité est une donnée géographique qui décrit les conditions d'accès d'un lieu au regard des nécessités des personnes ayant des besoins spécifiques, comprenant les personnes en situation de handicap, mais également les personnes avec une poussette ou un bagage \cite{Ding2014}. L'espace urbain dispose d'infrastructures pour permettre à une \gls{pcdv} de se repérer et de confirmer le cheminement qu'elles suit. À ce titre, en France, le \gls{cerema} produit des documents techniques à destination des aménageurs pour les guider sur les dispositifs existants et les bonnes pratiques pour les implanter (voir \ref{fig:ea_exemple_cerema_bev}). \comment{Référencer la partie de modélisation qui sera plus exhaustive sur le sujet}

\begin{figure}
    \centering
    \resizebox{\linewidth}{!}{\includegraphics{images/ea_exemple_cerema_bev.png}}
    \caption{Exemple issu de \cite{CEREMA2022} sur les bonnes pratiques d'implation de \gls{bev}.}
    \label{fig:ea_exemple_cerema_bev}
\end{figure}

% Parler des modèles de données qui permettent de contenir des informations sur l'accessibilité
% Netex, Géostandard, CityGML

\newpar{}

% Je reprécise 
% Je cite les principaux modèles (note de bas de page pour citer d'autres modèles -CityJSON-).
% Géostandard, NeTEx, GTFS, CityGML, OSM

Les modèles de données standardisés utilisés dans les domaines de l'urbanisme et du transport peuvent intégrer une dimension liée à l'accessibilité. En France, la \gls{lom}, promulguée en 2019, impose aux collectivités de collecter et publier les données d'accessibilité de la voirie à 200 mètres autour des arrêts de transports en commun. Le formalisme utilisé, le géostandard accessibilité normalisé par le \gls{cnig}, est un dérivé du \gls{netex} pensé pour la description fine des cheminements piéton. Le \gls{netex} est un format européen de description des réseaux de transports qui couvre un périmètre fonctionnel plus large que le géostandard du \gls{cnig}. Il permet déjà de représenter les cheminements piétons, mais avec une précision sémantique moindre. Il vise à supplanter en Europe la norme \gls{gtfs}, prévue pour le même usage, mais moins précise sur les questions d'accessibilité. D'autres formalisme plus généraux peuvent être mobilisés. CityGML \cite{Groeger2012}, pensé pour proposer une représentation 3D de l'espace urbain, est prévu pour être étendu à des usages spécifiques \cite{Biljecki2018}. Cette possibilité est exploitée par \cite{Wheeler2020} pour y intégrer les cheminements piéton et leurs conditions d'accessibilité.

L'émergence de l'\gls{igv} a apporté une nouvelle perspective à la collecte de données, permettant d'augmenter leur disponibilité sur des zones et des thèmes auparavant peu couverts \cite{Goodchild2007}, dont l'accessibilité. \gls{osm} est une base de données géographique généraliste, libre et collaborative couvrant une échelle mondiale. Son modèle repose sur un système de clés-valeurs\footnote{Sur \gls{osm}, une clé est nommée tag.} souple, dont les associations permettent de décrire une très grande diversité d'objets. L'utilisation des clés et leur valeur associée pour un objet est discutée et documentée sur le wiki du projet et, bien qu'il n'existe aucun mécanisme pour contraindre l'utilisation de clés spécifiques, des consensus existent pour de nombreux types entités.

\subsection{Les bases de données existantes}

L'acquisition des données géographiques est un processus généralement mené par des professionnels. Cela est également vrai pour les données d'accessibilité pour lesquelles la précision demandée nécessite une opération manuelle particulièrement longue et des relevés sur le terrain \cite{Beale2006}. % Je parle ici des BDD créées par les professionnels

% Je parle ici des BDD issues des données collaboratives

% Parler de leur représentation dans les BDD et de leur qualité : cette bdd modélise les choses comme ça et couvre ce territoire. Tableau comparatif.
% Temporalité : travaux
% Cycle des feux : on sait le modéliser mais la donnée n'existe pas. 

% Justification utilisation OSM : plein de paradigmes de modélisation de cartographie fine (CityGML, Netex, etc.). Le plus fexible pour nos besoins = OSM.

\todo{}

\section{Les dispositifs pour rendre accessibles les données géographiques}

% Dans cette partie, chaque sous section se termine avec un focus sur l'existant au sujet des carrefours.
% Introduire les données géographiques et les bases de données les plus communes (OpenStreetMap)

Les données géographiques peuvent être utilisées pour concevoir des cartes, mais également au sein de dispositifs sensoriels permettant à une \gls{pcdv} de les explorer. Pour cela, il est nécessaire d'en proposer une représentation accessible par d'autres sens que la vue. Cette partie s'intéresse à présenter les dispositifs décrits au sein de la littérature, ainsi que ceux effectivement mobilisés dans leur quotidien par les personnes concernées.

\subsection{Les descriptions autosuffisantes}

% Définition : une description autosuffisante c'est un texte mobilisable tel quel. Il n'y a pas d'interactions entre l'utilisateur et le texte.

% Une description verbale/textuelle aide à la construction d'une carte mentale

Une carte mentale construite à partir d'une description verbale partage ses propriétés avec une carte mentale dérivée d'une expérience visuelle, dont les relations spatiales entre les entités \cite{Denis1992}. Par ailleurs, elle permet d'inférer des relations spatiales non-explicitement indiquées dans la description, et ainsi d'adopter un point de vue exocentré \cite{Avraamides2004}. Une description verbale peut ainsi être mobilisée par les \pcdvs{} afin de construire une représentation mentale d'un espace.

\newpar{}

% Concevoir une description pour une PCDV

La littérature se concentre principalement sur la description d'itinéraires. Du point des sciences cognitives, de nombreux travaux se sont intéressés aux principes et pratiques pour communiquer efficacement la connaissance d’un itinéraire, en analysant les éléments qui le constituent et influent sur sa qualité \cite{Allen2000,Lovelace1999}. Les recherches qui se focalisent sur la question de la déficience visuelle abordent la description d'itinéraires en fonction des environnements intérieurs ou extérieurs.

Un modèle pour la création de description d'itinéraires en intérieur est proposé par \cite{Troeger2020}. Il repose sur des descriptions validées par des utilisateurs concernés et comprend plusieurs séquences de directions augmentées de repères. Une approche similaire est abordée par \cite{Kulyukin2008} pour le guidage au sein d'un commerce, dont les descriptions sont générées automatiquement à l'aide d'un graphe topologique représentant ce dernier.

Pour le guidage extérieur, \cite{gaunet_verbal_2006} décrit une suite d’instructions de guidage auxquelles sont associées une forme verbale, permettant de décrire un cheminement au sein de la ville. Celui-ci peut notamment comprendre une indication d’orientation angulaire ou relative à une voie de circulation. Une autre grammaire de description est élaborée par \cite{Constantinescu2019}. \cite{Balata2016} proposent une approche similaire et génèrent automatiquement les descriptions à l'aide d'un graphe piéton enrichi par des repères. Chaque instruction est basée sur un patron et déclenchée selon le type de segment parcouru.

% Quels sont les dispositifs proposant des descriptions pour le guidage (plutôt dans la partie in situ ?)

% Ce qui existe pour les carrefours

\newpar{}

Il existe peu de littérature visant spécifiquement la manière de décrire un carrefour. On peut cependant citer les travaux de \cite{gaunet_verbal_2006} qui intègrent au sein de l'itinéraire une description basique des carrefours, indiquant leur nombre de branches ainsi que l'orientation de chacune d'elle (voir \ref{fig:exdesccarrefour}).

\begin{figure}
    \centering
    \includegraphics{images/placeholder.jpg}
    \caption{La description des carrefours proposée par \cite{gaunet_verbal_2006} est intégrée à l'itinéraire. Les branches du carrefour sont indiquées.}
    \label{fig:exdesccarrefour}
\end{figure}

La génération automatique de description de carrefours est adressée par \cite{Guth2019}. En se basant sur la littérature en instruction et mobilité, ils proposent un schéma de base de données non relationnel contenant les principaux attributs permettant de décrire le carrefour, sa forme et les équipements permettant de le traverser. La description générée n'est pas phrasée mais verbalisée sous la forme de séquence d'éléments. Par ailleurs, la base de donnée n'est pas dérivée de données réelles.

\subsection{Les cartes en relief}

Les cartes en reliefs (voir \ref{fig:excartestactiles}) permettent d'explorer une représentation de la donnée géographique accessible au toucher. Elles sont souvent réalisées manuellement par les \ipas{} et les \adts{} (voir \ref{pratiques_ipas}). La revue de littérature de \cite{Wabinski2022} rappelle qu'elles répondent à des règles de conception précises afin de permettre à leur lecteur de les comprendre et d'en distinguer les différents éléments.

\begin{figure}
    \centering
    \includegraphics{images/placeholder.jpg}
    \caption{Exemples de cartes tactiles imprimées sur du papier thermogonflé.}
    \label{fig:excartestactiles}
\end{figure}

\newpar{}

La génération automatique de cartes en relief basées sur des données géographiques présente de nombreux défis scientifiques et techniques. Une revue de littérature proposée par \cite{Wabinski2019} présente de manière exhaustive les différentes approches existantes. Si d'autres travaux ont adressé la génération de carte à petite échelle, l'état de l'art ci-dessous se concentre sur l'échelle de la rue.

La proposition de \cite{Miele2004}, TMAP, permet la réalisation à la demande de cartes tactiles des linéaires de rues. Un service interactif permet à l'utilisateur de définir une emprise et de personnaliser la conception, par exemple au niveau de l'échelle et des éléments labellisés. Ces options vont servir à styliser la donnée, puis la carte est réalisée à l'aide d'une embosseuse braille\footnote{Une embosseuse braille est une imprimante qui permet de retranscrire en relief des caractères brailles. Les machines étant coûteuses et bruyantes, l'embossage tend à être remplacé par le thermogonflage.}. Cependant, le linéaire de rues utilisé pour générer les cartes provient d'une base de données limitée au territoire américain. 

\cite{Minatani2010} proposent une approche équivalente sur le territoire japonais, en proposant une infrastructure technique applicable à d'autres territoires sous réserve de disponibilité de données géographiques équivalentes. La carte est cette fois réalisée par thermogonflage\footnote{Le thermogonflage consiste à imprimer un document en noir et blanc sur un papier spécial contenant des microcapsules. Ces dernières gonflent de manière uniforme sur les zones imprimées lorsque le papier est passé au four.}. Ils insistent par ailleurs sur l'intérêt des repères sensoriels, qui peuvent différer selon l'utilisateur, ce qui implique d'en permettre la personnalisation. 

Les dispositifs précédents sont étendus par \cite{Watanabe2014, Cervenka2016} en utilisant la base de données OpenStreetMap comme source afin de permettre la génération de carte tactile sur n'importe quel territoire.

\newpar{}

Les travaux évoqués jusqu'ici reposent sur une stylisation de la donnée géographique. Cette dernière n'a cependant pas été pensée pour une représentation tactile et peut présenter des formes complexes difficiles à discerner. \cite{Stampach2016} s'appuient sur les approches précédentes en insistant sur l’intérêt du processus de généralisation pour la bonne distinction des différents éléments qui composent la carte. Les processus de généralisation pouvant donner un résultat inadéquat, une édition manuelle de la carte reste possible.

\cite{Touya2019} proposent d'aller au-delà de la généralisation pour "schématiser" la carte, c'est-à-dire en proposer une représentation abstraite plus adaptée aux zones urbaines denses. 

\newpar{}

Les bases de données vectorielles, bien qu'ayant parfois une couverture mondiale, n'assurent pas une exhaustivité sur leur échelle entière sur le thème de l'accessibilité. L'imagerie aérienne, en fonction de sa résolution, peut permettre de distinguer ces éléments. \cite{FillieresRiveau2020} s'intéressent à cette problématique en proposant de générer des cartes en relief par apprentissage profond en utilisant à la fois une photographie aérienne et un filaire de rues vectoriel. 

\newpar{}

Le sujet de la génération de cartes tactiles à l'échelle spécifique du carrefour est adressé au sein du projet ACTIVmap par \cite{Jiang2023} en utilisant les données OpenStreetMap et en accordant une attention particulière aux échelles et niveaux de détails requis.

Les travaux de génération de cartes tactiles reposent sur des techniques différentes. Ils ne s'intéressent pas à la même échelle ni à la même emprise. et les supports de réalisation visés sont également variés (voir \ref{tab:ea_relief}). Dans tous les cas, la carte en relief seule ne permet pas de représenter une haute densité d'informations \cite{Touya2019}, et est plus abordable lorsqu'elle est complétée d'une information sonore \cite{Brock2015}. Ce sujet est abordé dans \ref{ea_cartessonores}.

\begin{table}
\begin{center}
\scriptsize
\begin{tabular}{ | l | l | l | l | l | l | l | }
    Publication & Technique & Généralisé & Automatisé & Couverture & Échelle & Support \tabularnewline
    \hline
    \cite{Miele2004} & Stylisation & Non & Complet & États-Unis & Rues & Embossage \tabularnewline
    \cite{Minatani2010} & Stylisation & Non & Complet & Japon & Rues & Thermogonflage \tabularnewline
    \cite{Watanabe2014, Cervenka2016} & Stylisation & Non & Complet & Mondiale & Rues & Thermogonflage \tabularnewline
    \cite{Stampach2016} & Stylisation & Oui & Partiel & Mondiale & Rues & Thermogonflage \tabularnewline
    \cite{FillieresRiveau2020} & Apprentissage profond & Non & Complet & Mondiale & Rues & Impression 3D \tabularnewline
    \cite{Jiang2023} & Stylisation & Oui & Partiel & Mondiale & Carrefours & Thermogonflage \tabularnewline
\end{tabular}
\end{center}
\caption{Résumé des points abordés pour chaque entrée de l'état de l'art sur les cartes tactiles.}
\label{tab:ea_relief}
\end{table}

% Illustrer l'intérêt d'une carte audiotactile VS carte tactile classique

\label{ea_cartetactile}

\subsection{Les cartes sonores et audiotactiles}

\label{ea_cartessonores}

Un état des lieux des techniques mobilisables pour sonoriser les données géographiques est proposé par \cite{Krygier1994}, qui distingue notamment les sons réalistes, qui peuvent correspondre à une description verbale ou un son imitant le réel, et les sons abstraits.

\newpar{}

% Les sons réalistes

Parmi les sons réalistes, on trouve notamment les paysages sonores. \cite{Porteous1985} définit un paysage sonore comme \pquote{l'environnement sonore global d'une région}. Il représente l'ambiance sonore telle qu'elle est réellement perçue en un lieu. La proposition de \cite{Josselin2016} s'appuie sur la notion de paysage sonore pour sonoriser une carte topographique en associant un son à chaque couleur et en modulant son volume selon sa dominance dans la carte. Le son produit peut ainsi représenter un paysage sonore. Une alternative sous la forme d'une composition musicale originale est également proposée et répond plutôt à la définition d'un son abstrait tel que proposé par \cite{Krygier1994}.

% Les sons abstraits

Ces derniers sont  modulés par des variables sonores telles que le volume ou la durée et peuvent être directement corrélés à des données. C'est l'approche choisie par \cite{Schito2018} qui proposent d'explorer les données continues telles que les Modèles Numériques de Terrain (MNT) en faisant correspondre leurs valeurs aux variables sonores.

\newpar{}

% La cartographie sonore multimodale

Les techniques évoquées ci-dessus visent à explorer des données par le son uniquement. Mais, comme le propose \cite{Krygier1994}, ce dernier peut être utilisé de concert avec un autre dispositif sensoriel pour proposer une représentation multimodale apportant une dimension supplémentaire à la visualisation, cette dernière pouvant être utilisée pour représenter une variable supplémentaire. C'est dans ce sens que la proposition de \cite{Bearman2010} modélise la précision positionnelle de bâtiments en établissant un son spécifique pour chaque valeur de cette variable. Une approche similaire de \cite{Foteinou2022} s'intéresse à la durée des incendies en modulant la longueur du son joué au survol d'une entité.

Ces dispositifs multimodaux utilisent une interface d'ordinateur standard pour permettre à l'utilisateur d'interagir avec la carte, rendant leur usage impossible par un utilisateur déficient visuel. Pour répondre à cette problématique, la littérature s'est intéressée à la sonorisation des cartes tactiles.

\newpar{}

% Les cartes audiotactiles

Les cartes dites audio-tactiles se sont développées sur la base des dispositifs en relief interactifs. On en trouve de nombreux exemples parmi lesquels \cite{Loetzsch1994} ou encore \cite{Landau2001}, basés sur une interface tactile reliée à un ordinateur et permettant d'interagir avec celui-ci à l'aide d'un programme spécifique.
Talking TMAP \cite{Miele2006} est une extension de TMAP \cite{Miele2004} permettant d'interagir avec la carte pour obtenir des informations sur l'élément touché par synthèse vocale et repose sur le matériel conçu par \cite{Landau2001}. L'émergence des écrans tactiles grand public a permis la conception d'outils plus accessibles \cite{Brock2015} comme DERi.

\subsection{Les dispositifs d'aide in situ}

% Cannes électroniques

% Balises de guidage, balises sonores

% GPS

\todo{}

\subsection{La réalité virtuelle et les maquettes interactives}

% Références :
% - The Sound of Being There: Audiovisual Cartography with Immersive Virtual Environments, Hruby, 2019

\todo{}

\subsection{Synthèse et conclusion}

% Contribution de la thèse
% Élements de contexte pour expliquer dans quel cadre s'inscrit le travail
% Question générale de représenter par du texte des éléments présents sous une autre forme.
% Est-ce que ce travail adresse tous les besoins ? S'il ne les adresse pas, est-ce que cette approche peut intéresser.

Résumé des modalités de chaque entrée de l'état de l'art (voir \ref{tab:ea_resume}).

\begin{table}
\begin{center}
\scriptsize
\begin{tabular}{ | l | l | l | l | l | l | }
    Production & Localisation & Interaction & Référentiel & Usage & Échelle \tabularnewline
    \hline
\end{tabular}
\end{center}
\caption{Résumé des modalités de chaque entrée de l'état de l'art.}
\label{tab:ea_resume}
\end{table}

\todo{}