\chapter{État de l'art}

\section{Se repérer dans l'espace urbain pour une personne déficiente visuelle}

%
% Qu'est-ce que la déficience visuelle ? Comment différencier malvoyants et non-voyants ? Qu'est-ce qu'un résidu visuel ?
%

La déficience visuelle correspond à la perte de tout ou partie de la vue. On quantifie la capacité à discriminer les objets à l'aide de l'indicateur d'acuité visuelle \comment{Et aussi le champ de vision}. Ce dernier est notamment utilisé par l'OMS pour définir la déficience visuelle, et au-delà la malvoyance et la cécité. Ainsi, on parle de malvoyance lorsque l'œil le plus performant a un score de 3/10\up{ème}, et de cécité lorsqu'il a un score inférieur à 1/20\up{ème}. Selon le degré de cécité, il peut subsister un résidu visuel qui correspond à une perception de la lumière ou du contour des objets. 

La sévérité de l'atteinte visuelle affecte la capacité des personnes à se déplacer en autonomie \cite{homere_2023}. Au-delà de l'atteinte visuelle, les handicaps associés peuvent également contribuer à rendre le déplacement plus difficile. En ce sens, les besoins peuvent varier d'une personne à une autre.

\subsection{Les mécanismes sensoriels mobilisés par les PSDV}

%
% Stratégies de déplacement (technique de la canne blanche, chien guide, écholocation)
%

Plusieurs stratégies sont mises en place par les personnes concernées lors de leur déplacement pour compenser l'absence de vision à l'aide des autres sens. La principale concerne la canne blanche, qui joue un rôle double. Elle permet en premier lieu de détecter les obstacles et objets présents sur son chemin. Mais elle joue également un rôle social en signalant aux autres usagers que la personne qui la porte est aveugle ou malvoyante. Par ailleurs, il existe des cannes blanches électroniques, équipées de capteurs pouvant détecter des obstacles à distance. Ces dernières sont cependant moins utilisées que leur variante traditionnelle. 

Mais la canne blanche n'est pas suffisante pour se repérer dans l'espace, son champ d'action se limitant à une surface réduite face à l'utilisateur, au contraire de l'ouïe qui va permettre à la personne d'appréhender l'espace en écoutant les sons de l'environnement urbain. Détecter un moteur à l'arrêt peut par exemple permettre de repérer un passage piéton en aval d'une voiture. Les dispositifs sonores en ville qui équipent certains feux vont permettre à la fois de repérer une traversée, mais également de la réaliser en sécurité. L'ouïe joue également un rôle fort dans l'écholocation, c’est-à-dire la capacité à utiliser les échos pour percevoir des objets. \todo{}

% Au-delà de l'écholocation, parler de proprioception

% Les outils de l'autonomie : carte mentale PCF, solutions techniques type GPS

\newpar{}

En cognition spatiale, une carte mentale correspond à un modèle conceptuel d'un espace et des objets qui le composent \addref{Fernandes, 2019}. Pour se déplacer en autonomie dans un environnement urbain, avoir conscience de la position relative des espaces connus pour pouvoir s'orienter. Un aspect important de cette carte mentale concerne les repères. Ceux-ci vont permettre, à l'aide des notions d'indices et de preuves \addref{Jouffrais, 201X}, de confirmer sa position et sa direction. Les repères peuvent faire appel au sens du toucher (un objet que l'on va sentir à la canne), de l'ouïe (une fontaine en fonctionnement), ou de l'odorat (en boulangerie ouverte). On peut également définir les repères à l'aide de trois attributs : Permanents, Caractéristiques et Fiables (PCF) \comment{Citer le chapitre du livre}. Ainsi, un repère qui respecte ces trois critères sera tout le temps mobilisables, au contraire d'une fontaine ou d'un commerce qui peuvent être intermittents. 

La carte mentale peut être alimentée par les déplacements habituels des personnes qui repèrent l'environnement. Mais elle peut également l'être à l'aide de différents dispositifs. Une carte tactile peut aider à comprendre un espace non familier. Mais les personnes peuvent également utiliser un GPS piéton qui pourra décrire en cours de route l'itinéraire, l'infrastructure et les points d'intérêt croisés sur le chemin.

% Parler des causes de la déficience visuelle (DMLA, accident) ? Pourquoi parle-t-on de handicaps associés (conclusion de la sous-partie) ?

\todo{}

\subsection{L'infrastructure existante pour se repérer en ville}

\label{ea_infrastructure}

L'espace urbain dispose d'infrastructures pour permettre aux \pcdvs de se repérer et de confirmer le cheminement qu'elles suivent. Bien qu'il n'existe aucune norme d'implantation \comment{à vérifier}, le Centre d'études et d'expertise sur les risques, l'environnement, la mobilité et l'aménagement (CEREMA) produit des documents techniques pour conseiller les aménageurs sur l'implantation d'une infrastructure efficace. Cependant, les équipements ne sont aujourd'hui pas présents sur l'intégralité du territoire et, surtout, peuvent être négligés et ainsi se muer en difficulté supplémentaire.

\todo{}

\subsection{Les pratiques des \ipas{}}

\label{pratiques_ipas}

L'Association Francophone des Instructeurs pour l'Autonomie des personnes Déficientes Visuelles (AFIADV) est issue de la fusion de l'Association des Instructeurs de Locomotion pour personnes Déficientes Visuelles (AILDV) et de l'Association des Instructeurs pour l'Autonomie dans la Vie Journalière des personnes Aveugles ou Déficientes Visuelles (AVJADV). Ces dernières formaient respectivement les Instructeurs de Locomotion (IL) et les Instructeurs en Autonomie de la Vie Journalière (IAVJ). L'AFIADV forme aujourd'hui des Instructeurs pour l'Autonomie des personnes Déficientes Visuelles (IA) dont la profession regroupe les deux expertises précédentes. Dans ce manuscrit, nous allons nous intéresser particulièrement à l'instruction en locomotion, aussi appelée instruction en orientation et mobilité. \comment{Indiquer qu'il y a peu de littérature francophone sur les pratiques des IL}

\begin{figure}
    \centering
    \includegraphics{images/placeholder.jpg}
    \caption{Lors d'une séance de locomotion, un \ipa{} réalise parfois une carte de situation en utilisant des aimants. Ce dispositif permet notamment une interaction avec l'apprenant qui peut les déplacer selon sa compréhension du lieu. Pour représenter une zone plus complexe et faire une carte plus exploratoire, les \ipas{} réalisent des cartes imprimées sur du papier thermogonflé. Ces cartes sont aujourd'hui réalisées en décalquant une photographie aérienne et complétées par connaissance du terrain, notamment sur les zones couvertes par la végétation.}
    \label{fig:il_carte}
\end{figure}

Dans ce contexte, le rôle de l'\ipa{} est d'accompagner la personne aveugle ou malvoyante dans la compensation de la perte de la vue par l'usage des autres sens, en apprenant à la personne à utiliser une canne blanche, à mobiliser des repères, et à découvrir l'infrastructure urbaine. Elle peut, pour cela, mobiliser des cartes en relief. S'il existe des outils de conception automatisée (voir \ref{ea_cartetactile}), la pratique de réalisation des cartes en relief est aujourd'hui majoritairement artisanale \comment{Erreur insertion figure ?!}.

Pour les carrefours spécifiquement, à l'instar des aménageurs, les \ipas{} mobilisent à la fois les typologies standards et les branches. \todo{}

\section{L'accessibilité dans les données géographiques}

\todo{}

\subsection{Que sont les données d'accessibilité ?}

\todo{}

\subsection{Les bases de données existantes}

% 

\todo{}

\section{Rendre accessible les représentations des données géographiques}

% Introduire les données géographiques et les bases de données les plus communes (OpenStreetMap)

Une donnée géographique est une donnée, en général tabulaire, à laquelle sont accolées des informations spatiales. Elles peuvent être utilisées pour concevoir des cartes, mais également au sein de dispositifs de guidage. Pour être mobilisées par des \pcdvs, il est nécessaire d'en proposer une représentation accessible par le biais du toucher ou du son.

\subsection{La conception des cartes en relief}

%
% Ajouter un tableau résumé à la fin 

Les cartes en reliefs (voir \ref{fig:excartestactiles}) sont un premier exemple de représentation accessible de la donnée géographique. Elles sont souvent réalisées manuellement par les \ipas{} et les \adts{} (voir \ref{pratiques_ipas}). La revue de littérature de \cite{Wabinski2022} rappelle qu'elles répondent à des règles de conception précises afin de permettre à leur lecteur de les comprendre et d'en distinguer les différents éléments.

\begin{figure}
    \centering
    \includegraphics{images/placeholder.jpg}
    \caption{Exemples de cartes tactiles imprimées sur du papier thermogonflé.}
    \label{fig:excartestactiles}
\end{figure}

La génération automatique de cartes en relief basées sur des données géographiques présente de nombreux défis scientifiques et techniques. Une revue de littérature proposée par \cite{Wabinski2019} présente de manière exhaustive les différentes approches existantes. Si d'autres travaux ont adressé la génération de carte à petite échelle, l'état de l'art ci-dessous se concentre sur l'échelle de la rue.

La proposition de \cite{Miele2004}, TMAP, permet la réalisation à la demande de cartes tactiles des linéaires de rues. Un service interactif permet à l'utilisateur de définir une emprise et de personnaliser la conception, par exemple au niveau de l'échelle et des éléments labellisés. Ces options vont servir à styliser la donnée, puis la carte est réalisée à l'aide d'une embosseuse braille\footnote{Une embosseuse braille est une imprimante qui permet de retranscrire en relief des caractères brailles. Les machines étant coûteuses et bruyantes, l'embossage tend à être remplacé par le thermogonflage.}. Cependant, le linéaire de rues utilisé pour générer les cartes provient d'une base de données limitée au territoire américain. 

\cite{Minatani2010} proposent une approche équivalente sur le territoire japonais, en proposant une infrastructure technique applicable à d'autres territoires sous réserve de disponibilité de données géographiques équivalentes. La carte est cette fois réalisée par thermogonflage\footnote{Le thermogonflage consiste à imprimer un document en noir et blanc sur un papier spécial contenant des microcapsules. Ces dernières gonflent de manière uniforme sur les zones imprimées lorsque le papier est passé au four.}. Ils insistent par ailleurs sur l'intérêt des repères sensoriels, qui peuvent différer selon l'utilisateur, ce qui implique d'en permettre la personnalisation. 

Les dispositifs précédents sont étendus par \cite{Watanabe2014, Cervenka2016} en utilisant la base de données OpenStreetMap comme source afin de permettre la génération de carte tactile sur n'importe quel territoire.

Les travaux évoqués jusqu'ici reposent sur une stylisation de la donnée géographique. Cette dernière n'a cependant pas été pensée pour une représentation tactile et peut présenter des formes complexes difficiles à discerner. \cite{Stampach2016} s'appuient sur les approches précédentes en insistant sur l’intérêt du processus de généralisation pour la bonne distinction des différents éléments qui composent la carte. Les processus de généralisation pouvant donner un résultat inadéquat, une édition manuelle de la carte reste possible.

\cite{Touya2019} proposent d'aller au-delà de la généralisation pour "schématiser" la carte, c'est-à-dire en proposer une représentation abstraite plus adaptée aux zones urbaines denses. 

Les travaux précédents se reposent sur des bases de données vectorielles. Cependant, même si la couverture des bases peut être mondiale, leur exhaustivité sur le thème de l'accessibilité n'est pas assurée sur tous les territoires. \cite{FillieresRiveau2020} s'intéressent à cette problématique en proposant de générer des cartes en relief par apprentissage profond en utilisant à la fois une photographie aérienne et un filaire de rues vectoriel. 

Au sein du projet ACTIVmap, \cite{Jiang2023} s'intéressent à la génération automatique de carte en relief depuis OpenStreetMap, en adressant l'échelle spécifique du carrefour et les niveaux de détails requis.

\newpar{}

Les travaux de génération de cartes tactiles reposent sur des techniques différentes. Ils ne s'intéressent pas à la même échelle ni à la même emprise. et les supports de réalisation visés sont également variés (voir \ref{tab:ea_relief}).

Dans tous les cas, la carte en relief seule ne permet pas de représenter une haute densité d'informations \cite{Touya2019}, et est plus abordable lorsqu'elle est complétée d'une information sonore \cite{Brock2015}.

% Revoir les colonnes, citer le tableau dans le texte.

\begin{table}
\begin{center}
\scriptsize
\begin{tabular}{ | l | l | l | l | l | l | l | }
    Publication & Technique & Généralisé & Automatisé & Couverture & Échelle & Support \tabularnewline
    \hline
    \cite{Miele2004} & Stylisation & Non & Complet & États-Unis & Rues & Embossage \tabularnewline
    \cite{Minatani2010} & Stylisation & Non & Complet & Japon & Rues & Thermogonflage \tabularnewline
    \cite{Watanabe2014, Cervenka2016} & Stylisation & Non & Complet & Mondiale & Rues & Thermogonflage \tabularnewline
    \cite{Stampach2016} & Stylisation & Oui & Partiel & Mondiale & Rues & Thermogonflage \tabularnewline
    \cite{FillieresRiveau2020} & Apprentissage profond & Non & Complet & Mondiale & Rues & Impression 3D \tabularnewline
    \cite{Jiang2023} & Stylisation & Oui & Partiel & Mondiale & Carrefours & Thermogonflage \tabularnewline
\end{tabular}
\end{center}
\caption{Résumé des points abordés pour chaque entrée de l'état de l'art sur les cartes tactiles.}
\label{tab:ea_relief}
\end{table}

% Illustrer l'intérêt d'une carte audiotactile VS carte tactile classique

\label{ea_cartetactile}

\subsection{La sonorisation des données géographiques}

Sonoriser la donnée géographique, de manière indépendante ou de concert avec un autre dispositif sensoriel, est un autre moyen de la rendre accessible. Plusieurs approches sont abordées par la littérature.

% Approche par son spécifique

La proposition de \cite{Josselin2016} consiste à sonoriser une carte topographique en associant un son à chaque couleur et en modulant son volume selon sa dominance dans la carte pour recréer un paysage sonore ou une composition spécifique.

\cite{Schito2018} proposent d'explorer les données continues telles que les Modèles Numériques de Terrain (MNT) en modulant la hauteur, la balance stéréo et la longueur d'un son.

Cependant, les techniques évoquées ci-dessus visent à explorer des données à petite échelle, elles ne sont donc pas adaptées aux données à l'échelle de la rue. Pour ces dernières, la littérature propose plutôt de verbaliser l'information.

\newpar{}

% Décrire une donnée générique : de la donnée vers le texte
% Utilisé dans plein de cas d'utilisation pour décrire des choses qui ne sont pas forcément liées à l'accessibilité

% L'aide à la navigation

L'aide à a navigation peut permettre en cours d'itinéraire d'obtenir des informations sur celui-ci pour s'orienter. Il peut être mobilisé à la fois par des personnes voyantes, et par des \pcdvs{} à travers un GPS par exemple. Les informations fournies lors de la navigation se doivent d'être claires pour être efficaces. La littérature s'est intéressée à la manière de décrire un itinéraire. \addref{Gaunet, 2006} propose ainsi une suite d’instructions verbales de guidage en zone urbaine.


\todo{}

\subsection{La description des carrefours}

% Décrire un carrefour

Il existe peu de littérature visant spécifiquement la description de carrefours. Les travaux de \addref{Gaunet, 2006} intègrent au sein de l'itinéraire une description basique des branches des carrefours. La génération automatique de description de carrefours est adressée par \addref{Guth et al., 2019}, qui propose un schéma de base de données permettant de décrire succinctement la configuration générale du carrefour et l'enchaînement des éléments de chaque traversée. Cette base de donnée est complétée manuellement, et non générée depuis des données géographiques.

\todo{}

\subsection{Contributions de la thèse}

% Élements de contexte pour expliquer dans quel cadre s'inscrit le travail
% Question générale de représenter par du texte des éléments présents sous une autre forme.
% Est-ce que ce travail adresse tous les besoins ? S'il ne les adresse pas, est-ce que cette approche peut intéresser.

\todo{}