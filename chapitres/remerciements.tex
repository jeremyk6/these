\chapter*{Remerciements}

Mes premiers remerciements vont évidemment à mes directeurs de thèse, Fabien Feschet, Jean-Marie Favreau et Guillaume Touya. Merci de m'avoir accompagné pendant ces quatre années. Je suis sensible à l'environnement professionnel dans lequel j'évolue, et j'ai eu la chance une fois de plus de travailler avec des personnes humaines (même les informaticiens peuvent l'être !) et bienveillantes. Je n'aurais jamais pu aller au bout de ce travail sans votre soutien, vos conseils, et votre patience.

\newpar{}

Je remercie également les membres de mon jury de thèse, Anke Brock, Didier Josselin, François Pinet, Myriam Servières et Mathieu Simonnet qui ont accepté de consacrer du temps à l'évaluation de ce travail. Les perspectives que vous avez soulevé pendant la soutenance étaient enthousiasmantes, et j'espère que les outils que nous avons esquissé viendront à être développés dans les années à venir !

\newpar{}

Le projet ACTIVmap était composé de nombreuses personnes, réparties sur trois laboratoires et une entreprise: le LIMOS, l'IGN, l'IRIT et FeelObject. Merci à tous, c'était un plaisir de travailler et d'échanger avec vous. Je remercie en particulier les deux autres doctorantes du projet, Elen Sargsyan et Markie Jiang: on a tous réussi à en voir le bout !

\newpar{}

Ce travail est destiné à être utilisé par des personnes concernées et des professionnels de la déficience visuelle. Merci à Laurence, Nicolas, Sébastien, Manon, Céline et Didier pour votre temps. Vous m'avez accompagné dès mon arrivée en 2020, et je suis heureux d'avoir pu échanger avec vous sur vos besoins et vos attentes.

\newpar{}

L'exigence d'une thèse est bien entendu incompatible avec des <<~side projects~>>. Pour cela, je remercie Jean-Marie (encore), Samuel, Colas, Sébastien et les membres des Petits Débrouillards et d'OSM Clermont. J'ai adoré \sout{travailler} m'amuser avec vous sur le Pigeon Nelson, les balises sonores, les ateliers de cartographie, et j'en passe.

\newpar{}

Je ne me destinais pas particulièrement à la géomatique, mais j'ai eu la chance de croiser dans mon parcours des personnes bienveillantes, intéressantes et qui m'ont transmis le virus, et le goût d'une certaine qualité de vie au travail. Merci à Landry, Delphine, David, Philippe, et de manière générale au CRAIG et à la DDT de la Creuse. Merci également à mes collègues du SDIS de la Creuse, Hervé, Thierry, Philippe et Sébastien, ainsi qu'aux membres des autres services. 

\newpar{}

Merci aussi à la promo 2018-2020 du Master Géonum. Vous êtes de chouettes personnes, et c'était génial d'apprendre la géomatique avec vous pendant deux ans ! Un remerciement particulier à Léonie, Marie et Mathilde pour être venues assister à la soutenance !

\newpar{}

Merci également à mes amis, qui ont suivi ce travail de près ou de loin et que je suis heureux de retrouver à chaque fois (même si c'est évidemment trop peu souvent). Merci Rémi (qui est un excellent canard en caoutchouc depuis plus de dix ans), Max, Clément, Élise et Marien. Au groupe GTB, même si vous êtes relous, je vous aime quand même: merci à Pierre, Thibaut, Florian, Romain, Laura, Élodie et même Jordy !

\newpar{}

Enfin, je souhaite remercier ma famille. Mes parents, mes sœurs et mes grands-parents qui m'ont toujours soutenu dans mon parcours alambiqué et sur lesquels je sais pouvoir compter en toutes circonstances. Et bien entendu, merci à Ficelle qui ne lira jamais ce manuscrit, mais qui se devait d'être mentionnée !