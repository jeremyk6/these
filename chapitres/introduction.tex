\chapter*{Introduction}
\addcontentsline{toc}{chapter}{Introduction}

% mots-clés : inclusion, transport en commun, signalisation, outils (technologie, dispositifs)

% Contexte autour de la déficience visuelle

\section*{La déficience visuelle}

La déficience visuelle correspond à la perte de tout ou partie de la vue. Selon le degré de cécité, cela impacte l’autonomie des personnes dans leur quotidien et la mobilité en est un enjeu majeur. Elle nécessite alors des compétences particulières pour se repérer dans l'espace et analyser son environnement. Ces dernières peuvent s'acquérir auprès d'un \gls{ia}, qui accompagnera la personne pour l'aider à compenser l'absence de vision par l'usage d'autres sens comme l'ouïe ou le toucher. Lors de mises en situation, ou pour documenter un itinéraire, l'\gls{ia} pourra également réaliser une carte en relief du lieu étudié. Cette pratique est généralement artisanale et peut être réalisée à l'aide d'aimants, ou sur papier thermogonflé en décalquant une photographie aérienne.

% Le projet ACTIVmap et ses objectifs

\section*{Le projet ACTIVmap}

Le projet ACTIVmap\footnote{https://activmap.limos.fr}, pour Assistance à la Conception de carTes pour défIcients Visuels, est un projet ANR commencé en 2020 pour une durée de quatre ans. Il regroupe trois laboratoires aux spécialités complémentaires: le LASTIG en information géographique, l'IRIT en interaction homme-machine et le LIMOS en informatique, ainsi que l'entreprise FeelObject spécialisée dans la conception de cartes tactile audiodécrites. L'objectif du projet est de proposer des méthodes et outils d'aide à la génération de cartes tactiles mobilisables par les \glspl{ia}. Par ailleurs, le projet s'intéresse à l'adjonction d'audiodescriptions aux cartes pour augmenter leur portée et leur efficacité. Pour assurer la pérennité et la reproductibilité des réalisations, celles-ci se basent sur des données sous licence libre. L'espace d'expérimentation choisi se concentre sur les carrefours urbains. Ils représentent, à l'instar des espaces ouverts, une difficulté importante dans le déplacement des \glspl{pcdv} en ville par leur potentielle complexité. Il est ainsi nécessaire d'étudier attentivement la configuration et l'équipement d'un carrefour pour pouvoir le traverser en sécurité.

% Comment cette thèse s’inscrit dans le projet

\newpar{}

Cette thèse s'inscrit dans le projet en s'intéressant spécifiquement à la génération automatique d'audiodescriptions de carrefour depuis des données géographique. L'approche choisie est pluridisciplinaire: elle mobilise des connaissances et outils d'acquisition et de traitement de données issues de la géomatique, et de formalisation issues de l'informatique.

\section*{Organisation du manuscrit}

Le premier chapitre consacré à l'état de l'art abordera la manière dont les \glspl{pcdv} se repèrent dans l'espace urbain, en s'intéressant aux pratiques et dispositifs existants. Nous y définirons ce qu'est une donnée d'accessibilité, en examinant les bases existantes. Puis nous aborderons les travaux existants qui s'intéressent à la génération de représentations tactiles et textuelles de données géographiques, en particulier celles qui concernent les carrefours. Enfin, nous introduirons les contributions de cette thèse.

\newpar{}

Le second chapitre s'intéresse à la modélisation d'un carrefour urbain. Au sein de celui-ci, nous détaillons dans un premier temps une représentation graphe de la structure globale d'un carrefour. Celle-ci est abordée du point de vue d'un piéton tout en intégrant des éléments spécifiques à son accessibilité par les \glspl{pcdv}. Nous explorons ensuite les capacités d'\gls{osm} pour représenter un carrefour en mobilisant les contraintes spécifiées précédemment. En s'appuyant sur celles-ci, nous proposons des algorithmes permettant de segmenter un carrefour depuis le graphe d'\gls{osm}, et de l'instancier en un modèle objet facilitant sa manipulation.

\newpar{}

Le troisième chapitre présente différente modalités de description de données géographiques. Nous y présentons dans un premier temps un cadre général permettant la description textuelle de toute donnée géographique au sein d'un \gls{sig}. Puis nous présentons un cas d'usage de ce cadre appliqué à la description de carrefours. Nous déclinons cette approche en l'appliquant à plusieurs paradigmes de description.

\newpar{}

Le quatrième chapitre présente les expérimentations réalisées pour mettre en œuvre les méthodes décrites dans les chapitres précédents.
Nous y détaillons les implémentations des outils de segmentation et de modélisation de carrefours. Puis nous présentons un flux de travail intégré dans un \gls{sig} pour la génération automatique d'audiodescriptions de carrefours. Ce flux est appliqué à plusieurs paradigmes de description, figées et interactives, ainsi qu'à la contribution à la conception de cartes audiotactiles dans le cadre de preuves de concepts du projet ACTIVmap.

\newpar{}

Le cinquième chapitre se concentre sur l'évaluation des travaux réalisés. Dans un premier temps, nous présentons une évaluation des outils implémentés pour segmenter et modéliser les carrefours depuis \gls{osm}. Puis nous présentons une enquête à destination des \gls{ia} les invitant à modifier les audiodescriptions générées par nos outils pour qualifier des types de modifications à apporter aux descriptions. Enfin, nous mobilisons ces qualifications pour évaluer la possiblité de leur implémentation dans la chaîne de description. 

\newpar{}

Enfin, dans le chapitre de conclusion, nous réalisons une synthèse des travaux et contributions réalisés dans le cadre de cette thèse. Puis nous évoquons les perspectives de recherche et les travaux futurs qui pourraient être menés pour améliorer les méthodes et outils développés, avec un focus particulier sur la description au-delà des carrefours, et sur les apports possibles de l'intelligence artificielle.