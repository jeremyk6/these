\chapter{La modélisation du carrefour}

% Intro du chapitre : ma modélisation est basée sur OSM donc je décris ce que contient OSM

\todo{}

\section{Le carrefour pour une personne déficiente visuelle}

% Un "état de l'art" à la JLZ pour chaque sous-partie : une vue précise du carrefour, de son infrastructure et des éléments d'accessibilité qui le compose.
% Penser le carrefour du point de vue du piéton sous la forme d'un graphe

\todo{}

\subsection{La structure globale du carrefour}

% Structure en branches
% On se place du point de vue du piéton, ce sont les trottoirs qui comptent.

\todo{}

\subsection{Trottoirs et traversées : la structure en graphe du carrefour}

% Le graphe de circulation

\todo{}

\subsection{Les repères au sein du carrefour}

% PCF, obstacles

\section{La modélisation d’un carrefour au sein d’OpenStreetMap}

% Plein de paradigmes de modélisation de cartographie fine (CityGML, Netex, etc.). Le plus fexible pour nos besoins = OSM.

\todo{}

\subsection{Une forte capacité de représentation}

\todo{}

\subsection{Des variations dans les modélisations}

\todo{}

\section{Un modèle de carrefour}

\todo{}

\subsection{Une abstraction du carrefour en un modèle objet}

\todo{}

\subsection{Instancier le modèle depuis OpenStreetMap}

\todo{}